% Chapter 1
\selectlanguage{greek}
\chapter{Εισαγωγή} % Main chapter title

\label{Chapter1} % For referencing the chapter elsewhere, use \ref{Chapter1} 

\subsection*{Στερεοσκοπική όραση}
Στερεοσκοπική όραση ονομάζουμε τη μέθοδο που ακολουθούν πολλοί ζωντανοί οργανισμοί, ανάμεσα τους και ο άνθρωπος, για την απόκτηση τρισδιάστατης αντίληψης του χώρου στον οποίο βρίσκονται. Η στερεοσκοπική όραση προϋποθέτει την συνεργασία ενός αισθητηρίου οργάνου που είναι υπεύθυνο για την πρόσληψη της ορατής πληροφορίας (οφθαλμοί) και ενός τμήματος επεξεργασίας υπεύθυνο για την παραγωγή ουσιωδών συμπερασμάτων (εγκέφαλος). Το οπτικό σήμα (ηλεκτρομαγνητικό κύμα) προσπίπτει στην ίριδα του ματιού, διαθλάται στον κρυσταλλοειδή φακό ώστε να προσανατολιστεί κατάλληλα και καταλήγει στον αμφιβληστροειδή χιτώνα όπου μετατρέπεται σε νευρικό παλμό, μέσω των ραβδίων και των κονίων. Οι νευρικοί παλμοί μεταβιβάζονται στον εγκέφαλο μέσω των νευρικών κυττάρων του οπτικού νεύρου, όπου θα ακολουθήσει η κατάλληλη επεξεργασία. Ο εγκέφαλος μελετά την πληροφορία που έχει συλλεχθεί από τον κάθε οφθαλμό, δηλαδή τις δύο αμφιβληστροειδικές εικόνες, και συγκρίνει την οριζόντια διαφορά των απεικονιζόμενων αντικειμένων ανάμεσα στις δύο αποτυπώσεις. Όσο μεγαλύτερη απόκλιση εμφανίζεται τόσο κοντύτερα στους οφθαλμούς βρίσκεται το απεικονιζόμενο αντικείμενο και τούμπαλιν. Έτσι, αποκτάται αίσθηση του βάθους.

Σε πλήρη αναλογία λειτουργεί η στερεοσκοπική όραση στα τεχνητά υπολογιστικά συστήματα. Το ανάλογο των οφθαλμών είναι δύο ψηφιακές κάμερες σε στερεοσκοπική διάταξη και του εγκεφάλου το λογισμικό που επεξεργάζεται την πληροφορία του στερεοσκοπικού ζεύγους.

Σε αντίθεση με τους ανθρώπους, για τα τεχνητά υπολογιστικά συστήματα η στερεοσκοπική όραση δεν αποτελεί μονόδρομο για την αντίληψη του χώρου που τα περιβάλλει. Έχουν αναπτυχθεί τεχνικές βασισμένες σε εργαλεία όπως το \e lidar\footnote{\g σύμπτυξη των λέξεων \e light \g και \e radar \g} \g που απεικονίζουν άμεσα το τρισδιάστατο περιβάλλον, χωρίς την ενδιάμεση μετατροπή του σε δισδιάστατη πληροφορία (εικόνα) και την ακόλουθη ανακατασκευή του σε τρεις διαστάσεις. Αυτές οι τεχνικές όμως είναι ιδιαίτερα δαπανηρές και κατά περιπτώσεις μη εφαρμόσιμες, όπως για παράδειγμα όταν το υπό εξερεύνηση περιβάλλον είναι σε πολύ μεγάλη απόσταση (χαρτογράφηση περιοχών από αέρα, αποστολή \e STEREO \g της \e NASA \g κ.α.) ή περιέχει διαφανείς επιφάνειες (γυαλί, θάλασσα). Η στερεοσκοπική όραση παραμένει εξαιρετικά επίκαιρη μεθοδολογία, πολλές φορές σε αλληλοσυμπλήρωση με τις παραπάνω τεχνικές. 

Η στερεοσκοπική όραση στα υπολογιστικά συστήματα χωρίζεται σε δύο μεγάλες υποκατηγορίες. Ενεργή στερεοσκοπική όραση \e (active stereo vision) \g ονομάζεται όταν επιλύει το πρόβλημα με υποβοήθηση από κατάλληλα στοχευμένη εξωτερική πηγή δομημένου φωτός\footnote{παραδείγματα τεχνικών δομημένου φωτός: \e Conventional structured-light vision (SLV), Conventional active stereo vision (ASV), Structured-light stereo (SLS)} και παθητική \e (passive stereo vision) \g σε αντίθετη περίπτωση. Στην παρούσα εργασία ασχολούμαστε με την παθητική.

\subsection*{Μηχανική μάθηση}

Ως μηχανική μάθηση ορίζουμε τον τομέα της τεχνητής νοημοσύνης που δίνει τη δυνατότητα σε ένα υπολογιστικό σύστημα να μαθαίνει πως να περατώνει έναν σκοπό, χωρίς να έχει εκ των προτέρων προγραμματιστεί ρητά γι' αυτό.\footnote{Ορισμός κατά τον \e Arthur Samuel (1959).\g} Για την αυστηρότερη θεμελίωση του όρου "μαθαίνει", ο \e Tom Mitchell (1998) \g πρότεινε: "Όταν λέμε ότι ένα υπολογιστικό σύστημα μαθαίνει εννοούμε ότι από μια δεδομένη επίδοση \e P, \g με δεδομένη εμπειρία \e E \g σε ένα πρόβλημα \e T, \g η επίδοσή του \e P \g στο ίδιο πρόβλημα \e T \g βελτιώνεται καθώς αυξάνεται η εμπειρία του \e E". \g Ο όρος εμπειρία αναφέρεται στην ποσότητα παραδειγμάτων που έχει προσλάβει. Η διαδικασία της μάθησης ονομάζεται επιτηρούμενη ή επιβλεπόμενη \e (supervised) \g όταν το υπολογιστικό σύστημα δέχεται ως είσοδο παραδείγματα τα οποία εμπεριέχουν και την επιθυμητή έξοδο, λειτουργούν δηλαδή ως "δάσκαλος" και προσπαθεί μέσα από αυτά να δημιουργήσει έναν γενικό κανόνα πρόβλεψης κατάλληλης εξόδου ανάλογα με την είσοδο. Η μηχανική μάθηση χρησιμοποιείται όταν δεν είναι σαφές στον προγραμματιστή το "πως" ακριβώς περατώνεται ένας σκοπός.

Τα συνελικτικά νευρωνικά δίκτυα \e (convolutional neural networks) \g έχουν εμφανίσει εξαιρετικές επιδόσεις σε προβλήματα που λαμβάνουν εικόνα ως αρχική πληροφορία. Όπως περιγράφηκε παραπάνω, η στερεοσκοπική όραση απαιτεί την σύγκριση της σχετικής θέσης των προβεβλημένων αντικειμένων στις δύο λήψεις. Η σύγκρισή αυτή προϋποθέτει την απάντηση στο ερώτημα που βρίσκεται το ίδιο αντικείμενο στην κάθε λήψη. Η προσέγγιση του ερωτήματος με χρήση μηχανικής μάθησης αποδίδει αρκετά ποιοτικότερα αποτελέσματα.

Για την επίλυση του προβλήματος της υπολογιστικής στερεοσκοπικής όρασης συμπυκνώνεται γνώση από πολλά διαφορετικά επιστημονικά  πεδία. Χαρακτηριστικά αναφέρουμε:

\begin{itemize}
	\item \textbf{Φυσική:} Το φως είναι ηλεκτρομαγνητική ακτινοβολία συγκεκριμένου φάσματος\footnote{Το φάσμα του ορατού φωτός κυμαίνεται στο διάστημα $[400nm, 700nm]$} που διαδίδεται σε μορφή κύματος. Τα φαινόμενα που το περιγράφουν όπως η διάθλαση, η ανάκλαση και η διάχυση μελετώνται από την φυσική οπτική. 
	\item \textbf{Μαθηματικά:} Η προοπτική γεωμετρία ορίζει και περιγράφει τον σχηματισμό της εικόνας. Η γεωμετρία πολλαπλών προβολών μελετάει τους περιορισμούς που εισάγονται κατά την προβολή του ίδιου τοπίου σε πολλές λήψεις. Η γραμμική άλγεβρα και ο λογισμός πολλών μεταβλητών περιγράφουν δομές μηχανικής μάθησης, όπως τα τεχνητά νευρωνικά δίκτυα, και επιλύουν αποτελεσματικά προβλήματα βελτιστοποίησης. Η στατιστική κι η επεξεργασία σήματος μελετούν τα χαρακτηριστικά της εικόνας κι αναζητούν τρόπους εξαγωγής χρήσιμων συμπερασμάτων. 
	\item \textbf{Νευροεπιστήμες:} Το κύκλωμα διασυνδεδεμένων βιολογικών νευρώνων που αποτελεί τον νευρικό ιστό ενέπνευσε την δημιουργία των αντίστοιχων αλγορίθμων τεχνητών νευρωνικών δικτύων που αποτελούν βασικό εργαλείο της τεχνητής νοημοσύνης. Η τεχνητή νοημοσύνη αλληλεπιδρά αμφίδρομα με την ανθρώπινη, εμπνεόμενη από τους τρόπους με τους οποίους ο άνθρωπος λειτουργεί για να παράξει συμπέρασμα, γνώση, αντίληψη και σκέψη, αλλά παρέχοντας ταυτόχρονα εργαλεία ώστε να μελετηθούν αναλυτικότεροι οι τρόποι λειτουργίας του ανθρώπινου εγκεφάλου.
	\item \textbf{Επιστήμη Υπολογιστών:} Οι μέθοδοι αποτύπωσης της ηλεκτρομαγνητικές ακτινοβολίας σε ψηφιακή εικόνα, η ακόλουθη επεξεργασία της, η ανάπτυξη αλγορίθμων για την παραγωγή πληροφορίας και συμπερασμάτων από την ψηφιακή εικόνα, με ή χωρίς τη χρήση τεχνητής νοημοσύνης, η παραγωγή υλικού (αισθητήρες, επεξεργαστές, κάρτες γραφικών) για την ταχεία και ευσταθή περάτωση αυτών των αλγορίθμων είναι περιληπτικά κάποια από τα αντικείμενα της επιστήμης υπολογιστών.
\end{itemize}

\section{Παλαιότερες προσεγγίσεις στο πρόβλημα της στερεοσκοπικής όρασης}

\subsection*{Μέτρηση {ομοιότητας} χωρίων}

Η στερεοσκοπική όραση απαιτεί την αναγνώριση του απεικονιζόμενου αντικειμένου στις δύο λήψεις. Για την επίτευξη αυτού του σκοπού, συγκρίνουμε κάθε σημείο $(x,y)$ της εικόνας αναφοράς με όλα τα πιθανά σημεία που μπορεί να βρίσκεται στην έτερη λήψη και αποθηκεύουμε μια τιμή {ομοιότητας}. Το σύνολο των μετρήσεων αποθηκεύονται σε έναν τρισδιάστατο πίνακα:

$$C(d,x,y) = \texttt{ομοιότητα}[\texttt{εικόνα\_αναφοράς}(x,y), \texttt{έτερη\_εικόνα}(x-d,y)]$$

Οι πρώτοι αλγόριθμοι που υλοποιήθηκαν χρησιμοποίησαν ως μετρική σύγκρισης το "άθροισμα των απόλυτων διαφορών" \e (sum of absolute differences) \g \citep{anandan1989computational} \citep{matthies1989kalman}, το "άθροισμα των τετραγώνων των διαφορών" \e (sum of square differences) \g \citep{kanade1997development} και την κανονικοποιημένη ετεροσυσχέτιση ή ομοιότητα συνημιτόνου \citep{hannah1974computer}. Οι \e Zabih et. Woodfill (1994) \g \citep{zabih1994non} πρότειναν την μέθοδο \e Census \g που συγκρίνει τα περιφερειακά \e pixels \g του χωρίου με το κεντρικό, αποθηκεύοντας την τιμή $1$ αν είναι φωτεινότερα και $0$ αντίστροφα, δημιουργώντας έτσι μια δυαδική συμβολοσειρά από \e bits. \g Αυτή η συμβολοσειρά αποτελεί τον τοπικό περιγραφέα του χωρίου κι η μετρική σύγκρισης είναι ακολούθως η απόσταση \e Hamming \g των συμβολοσειρών. Οι \e Birchfield et. Tomasi (1998) \g \citep{birchfield1998pixel} συγκρίνουν κάθε \e pixel \g της εικόνας αναφοράς με μια συνάρτηση γραμμικής παρεμβολής της έτερη εικόνας. Τέλος, οι \e Mei et. al (2011) \g \citep{mei2011building} πρότειναν την μέθοδο \e AD-Census \g που συνδυάζει την πληροφορία από την σύγκριση μέσω "αθροίσματος απόλυτων διαφορών" και του μετασχηματισμού \e Census. \g

\subsection*{Τοπική άθροιση ομοιότητας}

Τοπικές μέθοδοι άθροισης εξομαλύνουν τις αρχικοποιημένες μετρήσεις {ομοιότητας} του προηούμενου βήματος. Η άθροιση ή ο υπολογισμός μέσης τιμής γίνεται σε υπολογισμένες περιοχές υποστήριξης, οι οποίες μπορεί να είναι είτε δισδιάστατες (χώρος $(x,y)$) είτε τρισδιάστατες (χώρος $(d,x,y)$). Οι πρώτες μέθοδοι που δοκιμάστηκαν εφάρμοζαν σε τετράγωνα χωρία φίλτρα μέσης τιμής. Οι \e Kanade et. Okutomi \g \citep{kanade1994stereo} και \e Kang et. Szeliski \g \citep{kang2001handling} πρότειναν την εφαρμογή μεταβλητών περιοχών υποστήριξης. Οι \e Zhang et. al \g \citep{zhang2009cross} πρότειναν τον υπολογισμό αυτών των περιοχών υποστήριξης με τη μέθοδο του σταυρού \e cross based cost aggregation, \g πετυχαίνοντας να μην εμπεριέχουν μεταβάσεις από ένα αντικείμενο σε ένα άλλο. Οι \e Scharstein et Szeliski \g \citep{scharstein1998stereo} πρότειναν την μέθοδο \e iterative diffusion \g που υπολογίζει σταθμισμένους μέσους όρους εντός των περιοχών υποστήριξης επαναληπτικά.

\subsection*{Υπολογισμός χάρτη παράλλαξης}

Σε αυτό το βήμα, με δεδομένο τον πίνακα $C$ μεταβαίνουμε στον πίνακα $D$ που περιέχει την οριζόντια μετατόπιση $d$ (ονομάζεται παράλλαξη) κάθε σημείου ανάμεσα στις δύο λήψεις. Η προφανής επιλογή συνίσταται στην επιλογή της παράλλαξης που εμφανίζει την μεγαλύτερη ομοιότητα, δηλαδή την εφαρμογή της πράξης $D = argmax_d(C)$.\footnote{Αν ο πίνακας C μετρούσε αντίθεση (κόστος) αντί για ομοιότητα, η αντίστοιχη πράξη θα ήταν $D = argmin_d(C)$.} Η παραπάνω λογική ονομάζεται \e winner takes it all. \g 

Έχουν προταθεί μέθοδοι που αντιμετωπίζουν τον πίνακα $D$ ως πρόβλημα καθολικής βελτιστοποίησης επιχειρώντας να δημιουργήσουν έναν λείο χάρτη παράλλαξης $D$ που λαμβάνει υπόψιν του τις τιμές ολόκληρου του πίνακα $C$. Αυτές οι μέθοδοι ορίζουν μια συνάρτηση ενέργειας: $$E_C(D) = E_{\texttt{ομοιότητας}}(D) + \tau E_{\texttt{εξομάλυνσης}}(D)$$ κι ακολούθως επιχειρούν να βρουν τον πίνακα παράλλαξης $D$ που ελαχιστοποιεί την τιμή του $E_C$.

Ο όρος $E_{\texttt{ομοιότητας}}(D)$ "προτιμά" τις παραλλάξεις με τις καλύτερες τιμές ομοιότητας:

$$E_{\texttt{ομοιότητας}}(D) = \sum_{\mathbf{p}} C(\mathbf{p},D(\mathbf{p}))$$

ενώ ο όρος $E_{\texttt{εξομάλυνσης}}(D)$ "προτιμά" την επιλογή ίδιων ή κοντινών τιμών παράλλαξης ανάμεσα σε γειτονικά σημεία:

$$E_\texttt{εξομάλυνσης}(D) = \sum_{\mathbf{p}} \sum_{\mathbf{q} \in N_p}g(D(\mathbf{p})-D(\mathbf{q}))$$

όπου $g$ μια γνησίως αύξουσα συνάρτηση.

Η καθολική εύρεση του ελάχιστου στο παραπάνω πρόβλημα, είναι υπολογιστικά αδύνατη. Προτάθηκαν μέθοδοι που αντιμετωπίζουν το πρόβλημα με πιθανοτικά γραφικά μοντέλα, όπως για παράδειγμα \e Markov Random Fields. \g Οι \e Boykov et. al (2001)\g \citep{boykov2001fast} και \e Kolmogorov et al. (2001) \g \citep{kolmogorov2001computing} αντιμετώπισαν το πρόβλημα με την μέθοδο \e graph cuts \g, ενώ οι \e Felzenszwalb et. al (2006) \g \citep{felzenszwalb2006efficient} πρότειναν την μέθοδο \e belief propagation. \g Ο \e Heiko Hirscmuller (2008) \g πρότεινε την μέθοδο \e Semi-Global Matching (SGM) \g \citep{hirschmuller2008stereo} που βρίσκει το ελάχιστο κατά μήκος $16$  προκαθορισμένων κατευθύνσεων μέσω δυναμικού προγραμματισμού. Έπειτα, υπολογίζει τον μέσο όρο των $16$ ελαχίστων, ως το ολικό ελάχιστο. 

\subsection*{Χρήση μηχανικής μάθησης}

Πριν την δημιουργία συλλογών στερεοσκοπικών δεδομένων με πληροφορία χάρτη παράλλαξης, λίγες προσεγγίσεις χρησιμοποιούσαν εκπαιδεύσιμα μοντέλα για την στερεοσκοπική αντιστοίχηση. Οι \e Kong, Tao (2004) \g \citep{kong2004method} εκπαίδευσαν ένα μοντέλο που αντιστοιχούσε σε κάθε αρχικό υπολογισμό ομοιότητας μια πιθανότητα: η πρόβλεψη να είναι σωστή, η πρόβλεψη να είναι λάθος λόγω αντικειμένου στο προσκήνιο, η πρόβλεψη να είναι λάθος για οποιονδήποτε άλλο λόγο. Έπειτα ακολουθούσε κατάλληλη επεξεργασία ανάλογα με την κατηγορία που άνηκε η κάθε πρόβλεψη.

Οι \e Zhang, Seitz (2007) \g \citep{zhang2007estimating}, οι \e Scharstein, Pal (2007) \g \citep{scharstein2007learning} και οι \e Li, Huttenlocher (2008) \g \citep{li2008learning} χρησιμοποίησαν μοντέλα για την εκμάθηση των βετιστων παραμέτρων των αντίστοιχων πιθανοτικών μοντέλων που χρησιμοποίησαν\footnote{\e markov random field \g και \e conditional random field}.

Οι \e Haeusler et al (2003) \g \citep{Haeusler_2013_CVPR} χρησιμοποίησαν έναν ταξινομητή \e random forest \g για την αξιολόγηση της ευστάθειας των αρχικών προβλέψεων ομοιότητας, ενώ οι \e Spyropoulos et. al (2014) \g \citep{Spyropoulos_2014_CVPR} χρησιμοποίησαν αυτές τις αξιολογήσεις για την ρύθμιση των παραμέτρων του \e markov random field \g που εφάρμοσαν στη συνέχεια. Σε αντίστοιχη λογική οι \e Park and Yoon (2015) \g \citep{park2015leveraging} χρησιμοποίησαν εκτιμήσεις ποιότητας των αρχικών προβλέψεων για την ρύθμιση του \e Semi-global matching. \g

\section{Προσεγγίσεις παρακείμενες στην προτεινόμενη μέθοδο}

Οι \e Zbontar and Lecun (2016) \g \citep{zbontar2016stereo} χρησιμοποίησαν ένα βαθύ νευρωνικό δίκτυο για την σύγκριση τετράγωνων περιοχών υποστήριξης κι αρχικοποιώντας κατά αυτό τον τρόπο τον πίνακα ομοιότητας. Εκπαίδευσαν το νευρωνικό δίκτυο σε τετράγωνα χωρία διάστασης $[9,9]$ \e pixels.\g Ακολούθως, εφάρμοσαν αρκετές τεχνικές για την βελτίωση των αποτελεσμάτων (όπως άθροιση κόστους, \e semi-global matching \g κ.α.). Οι μέθοδοι τους πέτυχαν κορυφαία αποτελέσματα για μεγάλο χρονικό διάστημα.

Οι \e Luo et. al (2016) \g \citep{Luo} ανέπτυξαν μια μέθοδο παρόμοια με αυτή των \e Zbontar and Lecun \g με δύο βασικές διαφορές. Εκπαίδευσαν το νευρωνικό δίκτυο με μεθόδους ταξινόμησης πολλαπλών κατηγοριών (το αντίστοιχο των \e Zbontar and Lecun \g είχε εκπαιδευτεί σε δυαδική ταξινόμηση), ενώ βελτίωσαν έντονα τους χρόνους εκτέλεσης επιλέγοντας αρχιτεκτονική λιγότερων παραμέτρων.

Oι \e Gidaris and Komodakis (2016)\g \citep{gidaris2016detect} υπολόγισαν έναν αρχικό χάρτη παράλλαξης μέσω του νευρωνικού δικτύου που εκπαίδευσαν οι \e Luo et. al \g κι ακολούθως χρησιμοποίησαν νέα επεξεργασία μέσω νευρωνικού δικτύου για την βελτίωση του αρχικού χάρτη παράλλαξης. Ουσιαστικά, προσπάθησαν να αντικαταστήσουν τις κλασικές μεθόδους άθροισης κόστους, \e semi-global matching, \g κλπ με μηχανική μάθηση. Το συνολικό μοντέλο τους χωρίζει το συνολικό πρόβλημα υπολογισμού του χάρτη παράλλαξης σε τρία μικρότερα υποπροβλήματα, καθένα εκ των οποίων λύνει μέσω νευρωνικών δικτύων.

Τέλος, οι \e Kendall et .al (2017) \g \citep{kendall2017end} εκπαίδευσαν ένα βαθύ νευρωνικό δίκτυο πολλών παραμέτρων, προσεγγίζοντας με μηχανική μάθηση το σύνολο της διαδικασίας υπολογισμού του χάρτη παράλλαξης. Η προσέγγισή τους αποτελεί αυτή τη στιγμή το \e state-of-the-art \g στην στερεοσκοπική συλλογή \e KITTI. \g

\section{Προτεινόμενη μέθοδος}

Στην παρούσα εργασία, αρχικά επιχειρούμε μια ποιοτική ανάλυση των αρχών και των περιορισμών της στερεοσκοπικής όρασης. Ακολούθως, επιλέγουμε συγκεκριμένα σημεία των παραπάνω μεθόδων και τα συνδυάζουμε σε μια ενιαία μεθοδολογία. Αναλύουμε σε ποιες βασικές στερεοσκοπικές αρχές βασίζεται η κάθε μεθοδολογία.

Αντιμετωπίζουμε το πρόβλημα της αρχικοποίησης του πίνακα ομοιότητας ως πρόβλημα ταξινόμησης πολλαπλών κατηγοριών. Κάθε σημείο της εικόνας αναφοράς οφείλει να κατηγοριοποιηθεί σε μια εκ του συνόλου των πιθανών οριζόντιων μετατοπίσεων. Εκπαιδεύουμε ένα βαθύ νευρωνικό δίκτυο ώστε να συγκρίνει τετράγωνα χωρία διάστασης $[19,19]$ \e pixels \g και να υπολογίζει μια πιθανοτική κατανομή ομοιότητας.

Στη συνέχεια, εφαρμόζουμε διαδοχικά βήματα ώστε να βελτιώσουμε τις αρχικές προβλέψεις ομοιότητας. Χρησιμοποιούμε την μέθοδο άθροισης κόστους σε περιοχές υποστήριξης υπολογισμένες με την μέθοδο του σταυρού (\e cross-based cost aggregation), \g προσπαθώντας να δημιουργήσουμε περιοχές υποστήριξης που θα περιορίζονται στην επιφάνεια ενός αντικειμένου. Έτσι, αποφεύγουμε την άθροιση πληροφορίας σε περιοχές μεταβάσεων που δημιουργούν μεγάλα σφάλματα.

Εφαρμόζουμε επίσης την μέθοδο \e semi-global matching \g με δυναμικό προγραμματισμό, περιορίζοντας τις διευθύνσεις βελτιστοποίησης σε $2$\footnote{Επιλύουμε το πρόβλημα σε $2$ διευθύνσεις και $2$ φορές ανά διεύθυνση. Έτσι έχουμε συνολικά $4$ διαφορετικές λύσεις από τις οποίες προκύπτει ο μέσος όρος.}, ώστε να μειωθεί η υπολογιστική πολυπλοκότητα και να επιταχυνθεί η εκτέλεση. 

Τέλος εφαρμόζουμε τις μεθόδους \e outlier detection \g και \e subpixel enhancement, \g όπως προτάθηκαν στην εργασία των \e Mei et al \g \citep{mei2011building}.

Αξιολογούμε το σύνολο της μεθόδου στις γνωστότερες συλλογές στερεοσκοπικών δεδομένων \e KITTI \g και \e Middlebury. \g Αποδεικνύουμε ότι η ακρίβεια της μεθόδου οφείλεται στην χρήση του νευρωνικού δικτύου και τις συγκρίσεις ομοιότητας που αυτό υλοποιεί και όχι στην ακόλουθη επεξεργασία. Η ακόλουθη επεξεργασία, αν και βελτιώνει αισθητά τα αποτελέσματα, είναι πολύ πιο αδύναμη αν δεχτεί ως είσοδο τον παραγόμενο πίνακα $C$ μιας συμβατικής μεθόδου σύγκρισης, όπως της μέσης απόλυτης διαφοράς. Αποδεικνύουμε επίσης ότι το νευρωνικό δίκτυο με κατάλληλη εκπαίδευση καταφέρνει να μάθει έναν ποιοτικό κανόνα σύγκρισης, που έχει επιτυχία ακόμα κι αν εφαρμοστεί σε εικόνες διαφορετικής στατιστικής από αυτές που εκπαιδεύτηκε.


\section{Δομή εργασίας}

Στο κεφάλαιο 2 παρουσιάζεται το θεωρητικό υπόβαθρο της στερεοσκοπικής μεθόδου. Αναλύονται οι βασικές αρχές στις οποίες βασίζεται η στερεοσκοπική μέθοδος, οι οποίες αποδεικνύονται μέσω της στερεοσκοπικής γεωμετρίας. Παρουσιάζονται οι υποθέσεις που κάνουμε για την στερεοσκοπική αντιστοίχηση και μελετώνται οι περιπτώσεις όπου αυτές οι υποθέσεις αίρονται. Γίνεται ανάλυση όλων των μεθόδων επεξεργασίας που χρησιμοποιούμε μετά την αρχικοποίηση του πίνακα ομοιότητας κι αναλύουμε σε ποιες υποθέσεις βασίζεται η κάθε μέθοδος.

Στο κεφάλαιο 3 επιλύουμε το πρόβλημα αρχικοποίησης του πίνακα ομοιότητας με χρήση μηχανικής μάθησης, μέσω τεχνητού νευρωνικού δικτύου. Αναλύουμε την αρχιτεκτονική του δικτύου που χρησιμοποιείται, γιατί επιλέχθηκε αυτή η αρχιτεκτονική, πως δημιουργούμε το σετ εκπαίδευσης και τις ακριβείς παραμέτρους εκπαίδευσης του δικτύου.

Στο κεφάλαιο 4 παρουσιάζονται τα αποτελέσματα της μεθόδου.

Τα κεφάλαια 2 και 3 έχουν τα αντίστοιχα παραρτήματά τους στο τέλος της εργασίας. Τα παραρτήματα αξιοποιούνται κυρίως για την απόδειξη των μαθηματικών σχέσεων και την αναλυτική παράθεση αλγοριθμικών και προγραμματιστικών τεχνικών. Επιλέξαμε αυτή την τακτική ώστε η γραφή μας εντός των κεφαλαίων να παραμένει προσηλωμένη στον κεντρικό στόχο κάθε ενότητας και να μην πλατειάζει στην ανάλυση ή την απόδειξη του κάθε εργαλείου που χρησιμοποιούμε. Παρ' όλα αυτά, για την σφαιρική κατανόηση των μεθόδων προτείνεται στον αναγνώστη να ανατρέχει στα παραρτήματα όπου υπάρχουν αναφορές σε αυτά.